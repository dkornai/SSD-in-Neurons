\documentclass[]{article}
\usepackage{geometry}
\geometry{a4paper, top=25mm, bottom=25mm, left=25mm, right=25mm}

%opening
\title{}
\author{}

\begin{document}

\maketitle

\begin{abstract}

\end{abstract}

\section{Ornstein-Uhlenbeck approximation}

\subsection{Introduction}

The changes in copy number of the system over time can be approximated as an Ornstein-Uhlenbeck process occurring in a given potential energy landscape. The change in copy number ($P$) is given by:

\begin{equation} \label{eq:o_u} 
	dP(t) = -\alpha \nabla U(P(t))dt + \sqrt{2 D} dW(t)
\end{equation} 

The gradient of the potential energy is defined for a given $P$ according to the expected change in $P$ due to the combined effects of actively controlled births, and deaths occurring with a constant rate:

\begin{equation} \label{eq:gradient} 
	\nabla U(P(t))dt = - P(max[0, Birth + C_b (NSS-P)] - Death)dt
\end{equation} 

Thus the full equation takes the form:

\begin{equation} \label{eq:o_u_full} 
	dP(t) = \alpha \times P(max[0, Birth + C_b (NSS-P)] - Death)dt + \sqrt{2 D} dW(t)
\end{equation} 

\subsection{Parameter estimation}

As the $Birth$, $Death$, $NSS$, and $C_b$ parameters are predetermined, simulating with the model requires the inference of the $D$ and $\alpha$ parameters. 

\subsubsection{Estimating D}
The estimation of $D$ is relatively straightforward. The Gillespie algorithm can be use to simulate a high number of replicates of the system with $C_b = 0$. With $C_b = 0$, copy numbers are no longer stabilized, and $P$ reverts to a simpler Brownian motion. For such 1-dimensional Brownian motion, the variance at time t is given by: 

\begin{equation} \label{eq:D} 
	Var(P(t)) = 2Dt
\end{equation} 
Therefore D can be estimated from the data observed during the Gillespie simulations as the best linear fit (with intercept 0) to the observed variance over time. 
\begin{equation} \label{eq:D_fit} 
	2D = \frac{\sum_{i} t_i \times \hat{Var(P(t))_i}}{\sum_{i} t_i^2}
\end{equation} 

\subsubsection{Estimating $\alpha$}
???


\end{document}
